\let\negmedspace\undefined
\let\negthickspace\undefined
\documentclass[journal]{IEEEtran}
\usepackage[a5paper, margin=10mm, onecolumn]{geometry}
%\usepackage{lmodern} % Ensure lmodern is loaded for pdflatex
\usepackage{tfrupee} % Include tfrupee package

\setlength{\headheight}{1cm} % Set the height of the header box
\setlength{\headsep}{0mm}     % Set the distance between the header box and the top of the text

\usepackage{gvv-book}
\usepackage{gvv}
\usepackage{cite}
\usepackage{amsmath,amssymb,amsfonts,amsthm}
\usepackage{algorithmic}
\usepackage{graphicx}
\usepackage{textcomp}
\usepackage{xcolor}
\usepackage{txfonts}
\usepackage{listings}
\usepackage{enumitem}
\usepackage{mathtools}
\usepackage{gensymb}
\usepackage{comment}
\usepackage[breaklinks=true]{hyperref}
\usepackage{tkz-euclide} 
\usepackage{listings}
% \usepackage{gvv}                                        
\def\inputGnumericTable{}                                 
\usepackage[latin1]{inputenc}                                
\usepackage{color}                                            
\usepackage{array}                                            
\usepackage{longtable}                                       
\usepackage{calc}                                             
\usepackage{multirow} 
\usepackage{hhline}                                           
\usepackage{ifthen}                                           
\usepackage{lscape}
\usepackage{circuitikz}
\tikzstyle{block} = [rectangle, draw, fill=blue!20, 
    text width=4em, text centered, rounded corners, minimum height=3em]
\tikzstyle{sum} = [draw, fill=blue!10, circle, minimum size=1cm, node distance=1.5cm]
\tikzstyle{input} = [coordinate]
\tikzstyle{output} = [coordinate]

\begin{document}
\bibliographystyle{IEEEtran}
\vspace{3cm}

\title{MatGeo Assignment 5.12.2}
\author{AI25BTECH11007}
 \maketitle
% \newpage
% \bigskip
{\let\newpage\relax\maketitle}

\renewcommand{\thefigure}{\theenumi}
\renewcommand{\thetable}{\theenumi}
\setlength{\intextsep}{10pt} % Space between text and floats


\numberwithin{equation}{enumi}
\numberwithin{figure}{enumi}
\renewcommand{\thetable}{\theenumi}
\noindent
\textbf{Question :}\\
Use elementary column operation $C_2 \to C_2 + 2C_1$ in the following matrix equation

\[
\begin{pmatrix}
2 & 1 \\
2 & 1
\end{pmatrix}
=
\begin{pmatrix}
3 & 1 \\
2 & 0
\end{pmatrix}
\begin{pmatrix}
1 & 0 \\
-1 & 1
\end{pmatrix}
\]

\noindent\\
\textbf{Solution :}
% We are given
% \[
% A = \begin{pmatrix}2 & 1\\[2pt]2 & 0\end{pmatrix}, \quad
% M = \begin{pmatrix}3 & 1\\[2pt]2 & 0\end{pmatrix}, \quad
% N = \begin{pmatrix}1 & 0\\[2pt]-1 & 1\end{pmatrix},
% \]
% so that
% \[
% A = M N.
% \]

% The elementary column operation $C_2 \to C_2 + 2C_1$ is represented by the elementary matrix
% \[
% E = \begin{pmatrix}1 & 2\\[2pt]0 & 1\end{pmatrix},
% \]
% because post-multiplying any matrix $X$ by $E$ leaves the first column of $X$ unchanged and replaces the second column by 
% \(\text{col}_2 + 2\,\text{col}_1\).

% Applying this to both sides:
% \[
% A E = M N E.
% \]

% Compute the left-hand side:
% \[
% A E =
% \begin{pmatrix}2 & 1\\[2pt]2 & 0\end{pmatrix}
% \begin{pmatrix}1 & 2\\[2pt]0 & 1\end{pmatrix}
% =
% \begin{pmatrix}2 & 5\\[2pt]2 & 4\end{pmatrix}.
% \]

% Compute the modified right factor:
% \[
% N E =
% \begin{pmatrix}1 & 0\\[2pt]-1 & 1\end{pmatrix}
% \begin{pmatrix}1 & 2\\[2pt]0 & 1\end{pmatrix}
% =
% \begin{pmatrix}1 & 2\\[2pt]-1 & -1\end{pmatrix}.
% \]

% Thus,
% \[
% M (N E) =
% \begin{pmatrix}3 & 1\\[2pt]2 & 0\end{pmatrix}
% \begin{pmatrix}1 & 2\\[2pt]-1 & -1\end{pmatrix}
% =
% \begin{pmatrix}2 & 5\\[2pt]2 & 4\end{pmatrix}.
% \]

% Hence the transformed matrix equation is
% \[
% \boxed{\;
% \begin{pmatrix}2 & 5\\[2pt]2 & 4\end{pmatrix}
% =
% \begin{pmatrix}3 & 1\\[2pt]2 & 0\end{pmatrix}
% \begin{pmatrix}1 & 2\\[2pt]-1 & -1\end{pmatrix}
% \;}.
% \]


We have
\[
A=\begin{pmatrix}2&1\\2&0\end{pmatrix}
=
\begin{pmatrix}3&1\\2&0\end{pmatrix}
\begin{pmatrix}1&0\\-1&1\end{pmatrix}.
\]

The column operation $C_2 \to C_2+2C_1$ is represented by the elementary matrix
\[
E=\begin{pmatrix}1&2\\0&1\end{pmatrix},
\]
since post-multiplication by $E$ performs the same column operation.

Thus, by matrix theory,
\[
AE = MNE,
\]
where
\[
AE=\begin{pmatrix}2&5\\2&4\end{pmatrix}, \qquad
NE=\begin{pmatrix}1&2\\-1&-1\end{pmatrix}.
\]

Hence,
\[
\boxed{\;
\begin{pmatrix}2&5\\2&4\end{pmatrix}
=
\begin{pmatrix}3&1\\2&0\end{pmatrix}
\begin{pmatrix}1&2\\-1&-1\end{pmatrix}
\;}.
\]

\end{document}