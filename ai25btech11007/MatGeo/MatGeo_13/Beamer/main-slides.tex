\documentclass{beamer}
\usepackage[utf8]{inputenc}
  
\usetheme{Madrid}
\usecolortheme{default}
\usepackage{amsmath,amssymb,amsfonts,amsthm}
\usepackage{txfonts}
\usepackage{tkz-euclide}
\usepackage{listings}
\usepackage{adjustbox}
\usepackage{array}
\usepackage{tabularx}
\usepackage{gvv}
\usepackage{lmodern}
\usepackage{circuitikz}
\usepackage{tikz}
\usepackage{graphicx}
\usepackage[T1]{fontenc}
\UseRawInputEncoding

\setbeamertemplate{page number in head/foot}[totalframenumber]

\usepackage{tcolorbox}
\tcbuselibrary{minted,breakable,xparse,skins}



\definecolor{bg}{gray}{0.95}
\DeclareTCBListing{mintedbox}{O{}m!O{}}{%
  breakable=true,
  listing engine=minted,
  listing only,
  minted language=#2,
  minted style=default,
  minted options={%
    linenos,
    gobble=0,
    breaklines=true,
    breakafter=,,
    fontsize=\small,
    numbersep=8pt,
    #1},
  boxsep=0pt,
  left skip=0pt,
  right skip=0pt,
  left=25pt,
  right=0pt,
  top=3pt,
  bottom=3pt,
  arc=5pt,
  leftrule=0pt,
  rightrule=0pt,
  bottomrule=2pt,
  toprule=2pt,
  colback=bg,
  colframe=orange!70,
  enhanced,
  overlay={%
    \begin{tcbclipinterior}
    \fill[orange!20!white] (frame.south west) rectangle ([xshift=20pt]frame.north west);
    \end{tcbclipinterior}},
  #3,
}
\lstset{
    language=C,
    basicstyle=\ttfamily\small,
    keywordstyle=\color{blue},
    stringstyle=\color{orange},
    commentstyle=\color{green!60!black},
    numbers=left,
    numberstyle=\tiny\color{gray},
    breaklines=true,
    showstringspaces=false,
}



\title 
{MatGeo Assignment 4.13.76}

\author
{AI25BTECH11007}
\begin{document}

\frame{\titlepage}
\begin{frame}{Question}
    Use elementary column operation $C_2 \to C_2 + 2C_1$ in the following matrix equation

\[
\begin{pmatrix}
2 & 1 \\
2 & 1
\end{pmatrix}
=
\begin{pmatrix}
3 & 1 \\
2 & 0
\end{pmatrix}
\begin{pmatrix}
1 & 0 \\
-1 & 1
\end{pmatrix}
\]
\end{frame}

\begin{frame}{Solution}
    We have
\[
A=\begin{pmatrix}2&1\\2&0\end{pmatrix}
=
\begin{pmatrix}3&1\\2&0\end{pmatrix}
\begin{pmatrix}1&0\\-1&1\end{pmatrix}.
\]

The column operation $C_2 \to C_2+2C_1$ is represented by the elementary matrix
\[
E=\begin{pmatrix}1&2\\0&1\end{pmatrix},
\]
since post-multiplication by $E$ performs the same column operation.

Thus, by matrix theory,
\[
AE = MNE,
\]
where
\[
AE=\begin{pmatrix}2&5\\2&4\end{pmatrix}, \qquad
NE=\begin{pmatrix}1&2\\-1&-1\end{pmatrix}.
\]
\end{frame}

\begin{frame}
Hence,
\[
\boxed{\;
\begin{pmatrix}2&5\\2&4\end{pmatrix}
=
\begin{pmatrix}3&1\\2&0\end{pmatrix}
\begin{pmatrix}1&2\\-1&-1\end{pmatrix}
\;}.
\]
\end{frame}

\begin{frame}[fragile]{C code}
    \begin{lstlisting}
        #include <stdio.h>

#define N 2

// Function to multiply two matrices
void multiply(int A[N][N], int B[N][N], int result[N][N]) {
    for (int i = 0; i < N; i++) {
        for (int j = 0; j < N; j++) {
            result[i][j] = 0;
            for (int k = 0; k < N; k++) {
                result[i][j] += A[i][k] * B[k][j];
            }
        }
    }
}
    \end{lstlisting}
\end{frame}

\begin{frame}[fragile]{C code}
    \begin{lstlisting}
        // Function to print matrix
void printMatrix(int A[N][N]) {
    for (int i = 0; i < N; i++) {
        for (int j = 0; j < N; j++) {
            printf("%d ", A[i][j]);
        }
        printf("\n");
    }
    printf("\n");
}

int main() {
    int A[N][N] = {{2, 1}, {2, 0}};
    int M[N][N] = {{3, 1}, {2, 0}};
    int Nmat[N][N] = {{1, 0}, {-1, 1}};
    int E[N][N] = {{1, 2}, {0, 1}};  // elementary matrix

    int AE[N][N], NE[N][N], MNE[N][N];
    \end{lstlisting}
\end{frame}

\begin{frame}[fragile]{C code}
    \begin{lstlisting}
        // Compute AE and NE
    multiply(A, E, AE);
    multiply(Nmat, E, NE);

    // Compute M * (NE)
    multiply(M, NE, MNE);

    printf("AE = \n");
    printMatrix(AE);

    printf("NE = \n");
    printMatrix(NE);

    printf("M * NE = \n");
    printMatrix(MNE);

    return 0;
}
    \end{lstlisting}
\end{frame}

\begin{frame}[fragile]{Python code}
    \begin{lstlisting}
        import numpy as np

# Define matrices
A = np.array([[2, 1], [2, 0]])
M = np.array([[3, 1], [2, 0]])
N = np.array([[1, 0], [-1, 1]])

# Elementary matrix for C2 -> C2 + 2*C1
E = np.array([[1, 2], [0, 1]])

# Compute AE, NE, and M * NE
AE = A @ E
NE = N @ E
MNE = M @ NE

print("AE =\n", AE)
print("\nNE =\n", NE)
print("\nM * NE =\n", MNE)
    \end{lstlisting}
\end{frame}
\end{document}