\let\negmedspace\undefined
\let\negthickspace\undefined
\documentclass[journal]{IEEEtran}
\usepackage[a5paper, margin=10mm, onecolumn]{geometry}
%\usepackage{lmodern} % Ensure lmodern is loaded for pdflatex
\usepackage{tfrupee} % Include tfrupee package

\setlength{\headheight}{1cm} % Set the height of the header box
\setlength{\headsep}{0mm}     % Set the distance between the header box and the top of the text

\usepackage{gvv-book}
\usepackage{gvv}
\usepackage{cite}
\usepackage{amsmath,amssymb,amsfonts,amsthm}
\usepackage{algorithmic}
\usepackage{graphicx}
\usepackage{textcomp}
\usepackage{xcolor}
\usepackage{txfonts}
\usepackage{listings}
\usepackage{enumitem}
\usepackage{mathtools}
\usepackage{gensymb}
\usepackage{comment}
\usepackage[breaklinks=true]{hyperref}
\usepackage{tkz-euclide} 
\usepackage{listings}
% \usepackage{gvv}                                        
\def\inputGnumericTable{}                                 
\usepackage[latin1]{inputenc}                                
\usepackage{color}                                            
\usepackage{array}                                            
\usepackage{longtable}                                       
\usepackage{calc}                                             
\usepackage{multirow} 
\usepackage{hhline}                                           
\usepackage{ifthen}                                           
\usepackage{lscape}
\usepackage{circuitikz}
\tikzstyle{block} = [rectangle, draw, fill=blue!20, 
    text width=4em, text centered, rounded corners, minimum height=3em]
\tikzstyle{sum} = [draw, fill=blue!10, circle, minimum size=1cm, node distance=1.5cm]
\tikzstyle{input} = [coordinate]
\tikzstyle{output} = [coordinate]

\begin{document}
\bibliographystyle{IEEEtran}
\vspace{3cm}

\title{MatGeo Assignment 5.5.24}
\author{AI25BTECH11007}
 \maketitle
% \newpage
% \bigskip
{\let\newpage\relax\maketitle}

\renewcommand{\thefigure}{\theenumi}
\renewcommand{\thetable}{\theenumi}
\setlength{\intextsep}{10pt} % Space between text and floats


\numberwithin{equation}{enumi}
\numberwithin{figure}{enumi}
\renewcommand{\thetable}{\theenumi}
\noindent
\textbf{Question:}\\
\[
\text{Using elementary row operations, find the inverse of the matrix }
A = \begin{bmatrix}
3 & -3 & 4 \\
2 & -3 & 4 \\
0 & -1 & 1
\end{bmatrix}
\]
\[
\text{and hence solve the following system of equations:}
\]
\[
\begin{cases}
3x - 3y + 4z = 21, \\
2x - 3y + 4z = 20, \\
- y + z = 5.
\end{cases}
\]

\noindent\\
\textbf{Solution :}
\[
\text{Let }\vec{A}=\begin{bmatrix}
3 & -3 & 4\\[4pt]
2 & -3 & 4\\[4pt]
0 & -1 & 1
\end{bmatrix}
\]

\[
\text{Augment } \vec{A} \text{ with the identity:}
\qquad
[\vec{A}\,|\,\vec{I}] =
\left(
\begin{array}{ccc|ccc}
3 & -3 & 4 & 1 & 0 & 0 \\[4pt]
2 & -3 & 4 & 0 & 1 & 0 \\[4pt]
0 & -1 & 1 & 0 & 0 & 1 \\
\end{array}
\right)
\]

Row transformation 1:\quad $R_2 \to 3R_2 - 2R_1$
\[
\left(
\begin{array}{ccc|ccc}
3 & -3 & 4 & 1 & 0 & 0 \\[4pt]
0 & -3 & 4 & -2 & 3 & 0 \\[4pt]
0 & -1 & 1 & 0 & 0 & 1
\end{array}
\right)
\]

Row transformation 2 :\quad $R_3 \to 3R_3 - R_2$
\[
\left(
\begin{array}{ccc|ccc}
3 & -3 & 4 & 1 & 0 & 0 \\[4pt]
0 & -3 & 4 & -2 & 3 & 0 \\[4pt]
0 & 0 & -1 & 2 & -3 & 3
\end{array}
\right)
\]

Row transformation 3 :\quad $R_3 \to -R_3$
\[
\left(
\begin{array}{ccc|ccc}
3 & -3 & 4 & 1 & 0 & 0 \\[4pt]
0 & -3 & 4 & -2 & 3 & 0 \\[4pt]
0 & 0 & 1 & -2 & 3 & -3
\end{array}
\right)
\]

Row transformation 4 and 5 :\quad
$R_2 \to R_2 - 4R_3,\quad R_1 \to R_1 - 4R_3$
\[
\left(
\begin{array}{ccc|ccc}
3 & -3 & 0 & 9 & -12 & 12 \\[4pt]
0 & -3 & 0 & 6 & -9 & 12 \\[4pt]
0 & 0 & 1 & -2 & 3 & -3
\end{array}
\right)
\]

Row transformation 6 :\quad Scale rows to get leading 1's:
\[
R_2 \to -\tfrac{1}{3}R_2,\qquad R_1 \to \tfrac{1}{3}R_1
\]
\[
\left(
\begin{array}{ccc|ccc}
1 & -1 & 0 & 3 & -4 & 4 \\[4pt]
0 & 1 & 0 & -2 & 3 & -4 \\[4pt]
0 & 0 & 1 & -2 & 3 & -3
\end{array}
\right)
\]

Row transformation 7 :\quad $R_1 \to R_1 + R_2$
\[
\left(
\begin{array}{ccc|ccc}
1 & 0 & 0 & 1 & -1 & 0 \\[4pt]
0 & 1 & 0 & -2 & 3 & -4 \\[4pt]
0 & 0 & 1 & -2 & 3 & -3
\end{array}
\right)
\]

\[
\text{Thus }
\boxed{\,\vec{A}^{-1} =
\begin{bmatrix}
1 & -1 & 0\\[4pt]
-2 & 3 & -4\\[4pt]
-2 & 3 & -3
\end{bmatrix}\,}
\]

Solving system of equations,
\[
\text{We have } 
\vec{A}^{-1} =
\begin{bmatrix}
1 & -1 & 0 \\[4pt]
-2 & 3 & -4 \\[4pt]
-2 & 3 & -3
\end{bmatrix}, 
\quad
\vec{B} =
\begin{bmatrix}
21 \\[4pt]
20 \\[4pt]
5
\end{bmatrix}.
\]

\[
\vec{x} = \vec{A}^{-1}\vec{B} =
\begin{bmatrix}
1 & -1 & 0 \\[4pt]
-2 & 3 & -4 \\[4pt]
-2 & 3 & -3
\end{bmatrix}
\begin{bmatrix}
21 \\[4pt]
20 \\[4pt]
5
\end{bmatrix}
\]

\[
\vec{x} =
\begin{bmatrix}
1 \\[4pt]
-2 \\[4pt]
3
\end{bmatrix}
\quad \Longrightarrow \quad
\boxed{x=1,\; y=-2,\; z=3}
\]

\end{document}