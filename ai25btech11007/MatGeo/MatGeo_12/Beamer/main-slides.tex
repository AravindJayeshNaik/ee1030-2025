\documentclass{beamer}
\usepackage[utf8]{inputenc}
  
\usetheme{Madrid}
\usecolortheme{default}
\usepackage{amsmath,amssymb,amsfonts,amsthm}
\usepackage{txfonts}
\usepackage{tkz-euclide}
\usepackage{listings}
\usepackage{adjustbox}
\usepackage{array}
\usepackage{tabularx}
\usepackage{gvv}
\usepackage{lmodern}
\usepackage{circuitikz}
\usepackage{tikz}
\usepackage{graphicx}
\usepackage[T1]{fontenc}
\UseRawInputEncoding

\setbeamertemplate{page number in head/foot}[totalframenumber]

\usepackage{tcolorbox}
\tcbuselibrary{minted,breakable,xparse,skins}



\definecolor{bg}{gray}{0.95}
\DeclareTCBListing{mintedbox}{O{}m!O{}}{%
  breakable=true,
  listing engine=minted,
  listing only,
  minted language=#2,
  minted style=default,
  minted options={%
    linenos,
    gobble=0,
    breaklines=true,
    breakafter=,,
    fontsize=\small,
    numbersep=8pt,
    #1},
  boxsep=0pt,
  left skip=0pt,
  right skip=0pt,
  left=25pt,
  right=0pt,
  top=3pt,
  bottom=3pt,
  arc=5pt,
  leftrule=0pt,
  rightrule=0pt,
  bottomrule=2pt,
  toprule=2pt,
  colback=bg,
  colframe=orange!70,
  enhanced,
  overlay={%
    \begin{tcbclipinterior}
    \fill[orange!20!white] (frame.south west) rectangle ([xshift=20pt]frame.north west);
    \end{tcbclipinterior}},
  #3,
}
\lstset{
    language=C,
    basicstyle=\ttfamily\small,
    keywordstyle=\color{blue},
    stringstyle=\color{orange},
    commentstyle=\color{green!60!black},
    numbers=left,
    numberstyle=\tiny\color{gray},
    breaklines=true,
    showstringspaces=false,
}



\title 
{MatGeo Assignment 5.5.24}

\author
{AI25BTECH11007}
\begin{document}

\frame{\titlepage}
\begin{frame}{Question}
    \[
\text{Using elementary row operations, find the inverse of the matrix }
A = \begin{bmatrix}
3 & -3 & 4 \\
2 & -3 & 4 \\
0 & -1 & 1
\end{bmatrix}
\]
\[
\text{and hence solve the following system of equations:}
\]
\[
\begin{cases}
3x - 3y + 4z = 21, \\
2x - 3y + 4z = 20, \\
- y + z = 5.
\end{cases}
\]
\end{frame}

\begin{frame}{Solution}
    \[
\text{Let }\vec{A}=\begin{bmatrix}
3 & -3 & 4\\[4pt]
2 & -3 & 4\\[4pt]
0 & -1 & 1
\end{bmatrix}
\]

\[
\text{Augment } \vec{A} \text{ with the identity:}
\qquad
[\vec{A}\,|\,\vec{I}] =
\left(
\begin{array}{ccc|ccc}
3 & -3 & 4 & 1 & 0 & 0 \\[4pt]
2 & -3 & 4 & 0 & 1 & 0 \\[4pt]
0 & -1 & 1 & 0 & 0 & 1 \\
\end{array}
\right)
\]

Row transformation 1:\quad $R_2 \to 3R_2 - 2R_1$
\[
\left(
\begin{array}{ccc|ccc}
3 & -3 & 4 & 1 & 0 & 0 \\[4pt]
0 & -3 & 4 & -2 & 3 & 0 \\[4pt]
0 & -1 & 1 & 0 & 0 & 1
\end{array}
\right)
\]

Row transformation 2 :\quad $R_3 \to 3R_3 - R_2$
\end{frame}
\begin{frame}
\[
\left(
\begin{array}{ccc|ccc}
3 & -3 & 4 & 1 & 0 & 0 \\[4pt]
0 & -3 & 4 & -2 & 3 & 0 \\[4pt]
0 & 0 & -1 & 2 & -3 & 3
\end{array}
\right)
\]

Row transformation 3 :\quad $R_3 \to -R_3$
\[
\left(
\begin{array}{ccc|ccc}
3 & -3 & 4 & 1 & 0 & 0 \\[4pt]
0 & -3 & 4 & -2 & 3 & 0 \\[4pt]
0 & 0 & 1 & -2 & 3 & -3
\end{array}
\right)
\]

Row transformation 4 and 5 :\quad
$R_2 \to R_2 - 4R_3,\quad R_1 \to R_1 - 4R_3$
\[
\left(
\begin{array}{ccc|ccc}
3 & -3 & 0 & 9 & -12 & 12 \\[4pt]
0 & -3 & 0 & 6 & -9 & 12 \\[4pt]
0 & 0 & 1 & -2 & 3 & -3
\end{array}
\right)
\]

Row transformation 6 :\quad Scale rows to get leading 1's:
\end{frame}
\begin{frame}
\[
R_2 \to -\tfrac{1}{3}R_2,\qquad R_1 \to \tfrac{1}{3}R_1
\]
\[
\left(
\begin{array}{ccc|ccc}
1 & -1 & 0 & 3 & -4 & 4 \\[4pt]
0 & 1 & 0 & -2 & 3 & -4 \\[4pt]
0 & 0 & 1 & -2 & 3 & -3
\end{array}
\right)
\]

Row transformation 7 :\quad $R_1 \to R_1 + R_2$
\[
\left(
\begin{array}{ccc|ccc}
1 & 0 & 0 & 1 & -1 & 0 \\[4pt]
0 & 1 & 0 & -2 & 3 & -4 \\[4pt]
0 & 0 & 1 & -2 & 3 & -3
\end{array}
\right)
\]

\[
\text{Thus }
\boxed{\,\vec{A}^{-1} =
\begin{bmatrix}
1 & -1 & 0\\[4pt]
-2 & 3 & -4\\[4pt]
-2 & 3 & -3
\end{bmatrix}\,}
\]
\end{frame}
\begin{frame}  
Solving system of equations,
\[
\text{We have } 
\vec{A}^{-1} =
\begin{bmatrix}
1 & -1 & 0 \\[4pt]
-2 & 3 & -4 \\[4pt]
-2 & 3 & -3
\end{bmatrix}, 
\quad
\vec{B} =
\begin{bmatrix}
21 \\[4pt]
20 \\[4pt]
5
\end{bmatrix}.
\]

\[
\vec{x} = \vec{A}^{-1}\vec{B} =
\begin{bmatrix}
1 & -1 & 0 \\[4pt]
-2 & 3 & -4 \\[4pt]
-2 & 3 & -3
\end{bmatrix}
\begin{bmatrix}
21 \\[4pt]
20 \\[4pt]
5
\end{bmatrix}
\]

\[
\vec{x} =
\begin{bmatrix}
1 \\[4pt]
-2 \\[4pt]
3
\end{bmatrix}
\quad \Longrightarrow \quad
\boxed{x=1,\; y=-2,\; z=3}
\]
\end{frame}

\begin{frame}[fragile]{C code}
    \begin{lstlisting}
        #include <stdio.h>

int main() {
    // Inverse of A
    double Ainv[3][3] = {
        {1, -1, 0},
        {-2, 3, -4},
        {-2, 3, -3}
    };

    // Vector b
    double b[3] = {21, 20, 5};

    // Result vector x = Ainv * b
    double x[3] = {0, 0, 0};
    \end{lstlisting}
\end{frame}

\begin{frame}[fragile]{C code}
    \begin{lstlisting}
        // Matrix-vector multiplication
    for (int i = 0; i < 3; i++) {
        for (int j = 0; j < 3; j++) {
            x[i] += Ainv[i][j] * b[j];
        }
    }

    // Print the solution
    printf("Solution of the system:\n");
    printf("x = %.2lf\n", x[0]);
    printf("y = %.2lf\n", x[1]);
    printf("z = %.2lf\n", x[2]);

    return 0;
}
    \end{lstlisting}
\end{frame}

\begin{frame}[fragile]{Python code}
    \begin{lstlisting}
        import numpy as np
# Inverse of A
A_inv = np.array([
    [1, -1, 0],
    [-2, 3, -4],
    [-2, 3, -3]
])

# Vector b
b = np.array([21, 20, 5])
# Solution x = A_inv * b
x = np.dot(A_inv, b)

print("Solution of the system:")
print(f"x = {x[0]}")
print(f"y = {x[1]}")
print(f"z = {x[2]}")
    \end{lstlisting}
\end{frame}

\end{document}